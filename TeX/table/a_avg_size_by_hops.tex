\begin{table}[!htb]
    \centering
\begin{tabular}{lrrrr}
\hline
        &   cover (\%) &     mean &      max &   std. deviation
 \\
\hline
 hop-1  &        0.01 &     4.10 &   177.00 &       2.56 \\
 hop-2  &        0.02 &    16.75 &   371.00 &      13.32 \\
 hop-3  &        0.06 &    49.09 &   775.00 &      39.79 \\
 hop-4  &        0.17 &   130.68 &  1683.00 &     112.06 \\
 hop-5  &        0.43 &   329.40 &  3596.00 &     291.45 \\
 hop-6  &        1.04 &   791.17 &  7091.00 &     700.34 \\
 hop-7  &        2.38 &  1806.73 & 13543.00 &    1560.90 \\
 hop-8  &        5.15 &  3910.70 & 23478.00 &    3242.12 \\
 hop-9  &       10.49 &  7965.52 & 38310.00 &    6206.44 \\
 hop-10 &       19.74 & 14981.34 & 55291.00 &   10607.87 \\
\hline
\end{tabular}
    \caption{the table describes how related articles of each articles covers in the entire corpus. The data is summarized from a list, where the number of related articles for every article in the corpus. \textit{mean}, \textit{max} and \textit{std. deviation} are the mean value, max value and standard deviation of the list respectively. \textit{cover} is computed from the value of \textit{mean} divided by the total amount of the articles}
    \label{tab:related_cover}
\end{table}
