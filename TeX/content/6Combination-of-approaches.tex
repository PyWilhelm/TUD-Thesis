\section{Combination of Approaches}
\label{sec:6}

1. Motivation:
上个section只是利用了text fields进行pure unsupervised的STS方法, 并没有利用meta-data和labeled数据中的related labels.因此 考虑使用supervised方法来将这三种信息combine得到最优的模型。 
这个过程可以抽象成一种Learning to Rank for Information Retrieval, (简述什么是LR)。LR有3种常用的分别为pointwise, pairwise和listwise,(分别描述是什么)。in our case, we choose pairwise. i.e. 认为任意两篇article生成的一条数据在训练和测试中是无差别的,训练阶段,regressor也不知道哪些数据是从同一个target生成而来的。


因此,在supervised中,training and testing data中,每个feature代表两篇articles在某一个方面的相关度,包括STS,processed meta-data, 而结果为这两篇是否相关的label,相关表示为1,不相关表示0. 对每一篇target来说,都可以产生Length(corpus) - 1条数据,数据总量为n*(length-1).

说明每个feature是如何计算的。
1. category
2. release date
3. keywords relevance 
4. ratio of term number
5. ratio of word number

如何计算最终的precision
对一个testing target,对每一篇在corpus中的文章生成一条数据,经过regressor之后得到一个probability of related article,然后选取概率最大的两条数据对应的文章作为最终的related articles. 然后计算平局precision与上个section方法一样。

在实验中,采用了几种可以regression的方法,分别为logistic regression, Bayesian regression 和XX。 

给出不同组合的结论得到最好的组合,并将最优模型带入实验二中,检测在reality中的实验结果是否一致。 
