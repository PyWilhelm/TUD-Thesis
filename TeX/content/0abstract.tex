\begin{abstract}
The purpose of the master thesis is to design an automated system to discover two related articles in the large corpora for a given target article. Each article contains the text fields including \ititle{}, \icontent{} and \isummary{} as the semantic information. The most important idea is that the score of \textit{Semantic Textual Similarity} (STS) between two articles is used for quantifying the probability or the degree of relatedness. Therefore, the STS methods, e.g. Jaccard Coefficient and Latent Semantic Indexing, are applied for building the semantic models to represent articles as semantic instances, such as vectors in the semantic vector space. The similarity is computed with comparison of the corresponding semantic instances and the candidate articles which have the highest similarity scores are judged as related to the target. However, quite a lot of information, particularly meta-data and the labeled relationships between articles, is not utilized in this way. In the viewpoint of information retrieval, the task can be treated as an application of learning to rank. The algorithms of learning to rank, such as linear regression and ListNet, are hence applied in the system. Instead of only the semantic information, each feature vector consists of the correlations of all kinds of information and indicates the entire relevance between the corresponding target and candidate. We find that the system using \tfidf{} in text field \icontent{} over unigram has the best performance with \textit{precision@2@3} of $45.07\%$ in terms of unsupervised approach. The system using the supervised method, linear regression, outperforms much better than the aforementioned unsupervised system. The precision of the best-performed system is improved to $64.0\%$. Besides, we also discuss the lack of the system and the reasons of the existing errors of false positive and false negative. The final system can be used in the scenario of reality in a semi-automated way. 
\end{abstract}